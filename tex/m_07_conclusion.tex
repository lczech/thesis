% ######################################################################################################################
%         Conclusion and Outlook
% ######################################################################################################################

\chapter{Conclusion and Outlook}
\label{ch:ConclusionOutlook}

\paperbox{
    This chapter is partially based on the peer-reviewed publication:
}{\paperpppp}{
    \textbf{Contributions:} Lucas Czech... Alexandros Stamatakis... \todo{fix all of them across the document!}
}


\todo{use tip edges as well: make them ``outside''. but would have to check implications of this inconsistency. would be different from both phylofactor and original def of edge imbalances}


\todo{general issue with placement: fixed tree means only one truth. what aobut different trees, eg baysian, or equally likeliy form a terrrass, etc. how can we use boostrap support etc? see washburn preprint, ll 865. true phylogeny is not knowable, hence it is estimated from data (ref to foundations chapter), inducing uncertainty}

\todo{gappa bionconda package}

In this thesis, we made several contributions to the field of computational phylogenetics,
and phylogenetic placement in particular.
% The methods were already described in two peer-reviewed publications \cite{Czech2018,Czech2018a},
% and the respective data and scripts were made available at \url{http://github.com/lczech/placement-methods-paper}.
We described the methods already in two peer-reviewed publications \cite{Czech2018,Czech2018a},
and made their data and scripts available at \url{http://github.com/lczech/placement-methods-paper}.
The implementation is available in our \toolname{genesis} library (\url{https://github.com/lczech/genesis});
we furthermore offer a ready-to-use command line interface for the presented methods
via our \toolname{gappa} tool (\url{https://github.com/lczech/gappa}).
This chapter briefly summarizes our contributions and gives an outlook for future development and research questions
in the field of phylogenetic placement.

Being first introduced between 2009 and 2010 \cite{Berger2009,Matsen2010a},
phylogenetic placement and its downstream analysis methods are still in their infancy,
but are becoming more and more popular in the research community.
Their advantages include greater interpretability and visualizability of results,
at the cost of additional methodological and computational complexity.
There are hence still obstacles to overcome before phylogeny-aware methods (such as phylogenetic placement)
can reach maturity and widespread use in the research community:
% generally less prevalent than simpler analysis methods,
% such as those based on sequence similarity.

\begin{itemize}
    \item \textbf{Usability:}
          Current software pipelines for phylogenetic placement are mostly build from separate programs and tools
          that are not well integrated with each other, and only bound together via intermediate files and scripts.
          This induces costs for first-time users of such tools, such as the manual compilation and setup of the software,
          as well as costs for ad-hoc scripting solutions to connect different steps of the pipeline.
    \item \textbf{Scalability:}
          The amount of sequence data is growing exponentially (see \secref{ch:Introduction:sec:Motivation}),
          currently doubling roughly every seven months \cite{Stephens2015}.
          Phylogeny-aware analysis methods for sequence data are generally more compute-intense than, e.\,g.,
          methods based on sequence similarity.
          The speed and scalability of these tools thus
          need to stay on a par with the amount of sequencing data being produced.
    \item \textbf{Analysis Methods and Research Questions:}
          Downstream analysis methods that take phylogenetic information into account
          are not yet as versatile and matured as classical data analysis methods.
          Hence, researchers might refrain from using phylogenetic placement and resort to established methodology.
          There are however many research questions that might benefit from analyses using phylogeny-aware methods,
          for example in disciplines such as bio-geography or health and medicine.
%           Methods are missing maybe because of the complexity.
\end{itemize}

We discuss our contributions to the field with respect to these obstacles in the following,
and describe potential research directions for overcoming them.

\paragraph{Usability and Scalability}
\label{ch:ConclusionOutlook:par:UsabilityScalability}

In \chpref{ch:AutomaticTrees}, we presented an approach to automatically obtain
reference trees for phylogenetic placement called \emph{\acfp{PhAT}} using large reference sequence databases.
We showed that \acp{PhAT} are valuable and accurate trees for
conducting phylogenetic placement and taxonomic assignment of metagenomic sequences.
They thus alleviate the labor-intense and potentially error-prone work of collecting suitable reference sequences,
and might even replace such ``manual'' trees in some applications.
% Hence, an interesting future direction might be to extend and evaluate them for other purposes and use cases such as ???
In the same chapter, we presented a multi-level placement approach as well as our pre-processing pipeline
to enable and speed up phylogenetic placement
of large, diverse datasets with hundreds to thousands of environmental sequence samples.
All these methods simplify and accelerate the processes for researchers
who want to apply phylogenetic placement and respective downstream analyses.

In order to further facilitate acquaintance with phylogenetic placement in the research community,
without the need for setup and scripting, online services for quickly testing standard methods might be worth to offer.
For example, uploaded query sequences could be placed on our four \acp{PhAT} used in the evaluation
(see \secref{ch:AutomaticTrees:sec:Evaluation:sub:ReferenceTreeSetup:par:SequenceSelection}),
or on custom reference trees.
These services could further offer standard analysis methods such as Edge~PCA and Squash Clustering
(\secref{ch:Foundations:sec:PhylogeneticPlacement:sub:ExistingMethods}),
or the methods that we presented in this thesis (see Chapters \ref{ch:Visualization} and \ref{ch:Clustering}).

For more advanced or large-scale studies, a custom setup is however usually inevitable,
both in order to scale up the processing using computer clusters as well as to allow customization of the workflow.
% State-of-the-art implementations of phylogenetic placement, such as our \toolname{EPA-ng} \cite{Barbera2018},
% are able to place billions of sequences withing a few hours on computer clusters.
% There are however other steps in typical pipelines that do not yet scale and hence form a bottleneck.
% For example, \toolname{PaPaRa} \cite{Berger2011a,Berger2012},
% a phylogeny-aware tool for aligning query sequences against a reference alignment,
% might benefit from a re-implementation.
Such studies might benefit from offering well-defined integrated pipelines for the basic steps of phylogenetic placement,
with a simple setup for all common functionality, for instance by using platforms such as
\toolname{Galaxy} \cite{Afgan2018} or \toolname{Conda}/\toolname{Bioconda} \cite{Gruning2018}.
Tools and pipelines that integrate different steps of the analysis can furthermore be implemented more efficiently,
as they can make better use of computing infrastructure such as \texttt{\acs{MPI}},
and distribute computations with less overhead for intermediate files and bookkeeping between pipeline steps.

% can be implemented in integrated ways for computer clusters.
% which will be necessary because of the large amounts of data that are coming our way

Moreover, to achieve a tighter interaction of the tools within a pipeline,
and to incentivize experimentation with existing methods and development of novel methods,
file formats need to be flexible and extensible.
For example, the \fileformat{jplace} file standard for phylogenetic placement \cite{Matsen2012}
is based on the \fileformat{JSON} format \cite{JsonMemo,JsonStandard}, and hence easily extensible.
The standard however currently (as of \fileformat{jplace} version 3) lacks support for multiple samples per file,
and additional per-sample or even per-query annotations and other meta-data.
A similar issue applies to file formats for phylogenetic trees, as we showed in \cite{Czech2017}.
In particular, the \fileformat{Newick} format is often extended ad-hoc for specific needs
(this is even done within the \fileformat{jplace} format itself),
with no standardized support from downstream tools or tree viewers.
Creating proper standards could thus also help to avoid ambiguities and misinterpretations of results.

% before researchers actually tune existing methods to their particular needs
% and start extending and experimenting with existing approaches.
% hinders rapid testing of methods,

With respect to the methods and tools presented in this thesis,
we plan to extend our \toolname{gappa} tool to include more existing analysis methods
and additional pre- and post-processing functions such as filtering, merging and manipulating placement files.
In the long term, we also want to re-implement all functionality offered by the
\toolname{guppy} tool from the \toolname{pplacer} suite \cite{Matsen2010},
and thereby make \toolname{gappa} a more efficient and scalable replacement for \toolname{guppy}.

Furthermore, we are planning to extend the \toolname{genesis} library, which is written in \texttt{C++11},
to offer \acs{API} bindings for \texttt{Python} as well.
Hence, \toolname{genesis} could combine the advantages of both programming languages:
For standard functions such as file processing and most internal computations,
it can use our highly efficient \texttt{C++} implementations,
which often make use of multi-threading via \texttt{OpenMP} (Open Multi-Processing) for multi-core computers.
The modular and clear \acs{API} of \toolname{genesis} further allows its use in production code.
At the same time, ad-hoc solutions for pipeline tasks such as file conversions or extraction of certain data
could be implemented as \texttt{Python} scripts.
This also facilitates to use \toolname{genesis} for experimentation and rapid prototyping of novel ideas and methods.

\paragraph{Analysis Methods}
\label{ch:ConclusionOutlook:par:AnalysisMethods}

% downstream methods for phylogenetic placement are not yes as plenty as standard sequence based methods

In \chpref{ch:Visualization}, we described methods for visualizing the phylogenetic placement of large metagenomic datasets.
The methods allow to detect differences between samples (\emph{Edge Dispersion}),
as well as correlations with per-sample meta-data (\emph{Edge Correlation}),
and thus are intended for similar use cases as the established Edge PCA \citep{Matsen2011a}.
However, our novel methods directly visualize important features of the samples (an their meta-data)
on the underlying reference tree, which allows interpretation in a phylogenetic context.
Furthermore, in \chpref{ch:Clustering}, we introduced clustering methods for metagenomic samples
(\emph{Phylogenetic $k$-means} and \emph{Imbalance $k$-means}),
which serve a similar purpose as Squash Clustering \citep{Matsen2011a},
but are better suited for larger datasets.
% The presented methods can also be combined, allowing

These novel methods extend the portfolio of available analysis tools for phylogenetic placement.
There are however still many approaches from the fields of machine learning and data mining
that could be useful to adapt to phylogenetic placement data.
It might further be worth to develop integrative methods that can incorporate heterogeneous features
such as phylogenetic data and different types of meta-data in a combined analysis \cite{Mariette2018}.
% e.\,g., using unsupervised machine learning methods \cite{Mariette2018}.
The challenge for adapting existing methods usually consists in making the methods phylogeny-aware,
for example by having them operate on distributions on trees
(including respective distance measures such as the KR distance) rather than on $\mathbb{R}^d$ vectors.

\todo{the following is no longer needed, or needs to be rephrased to fit with the new method!}

A further approach to contend with the compositional nature of metagenomic data are
the node-based balances using the isometric log-ratio of OTU abundances \cite{Silverman2017,Washburne2017a}
that we discussed in \secref{ch:Foundations:sec:PhylogeneticPlacement:sub:PlacementProcessing:par:EdgeImbalances}.
% methods related to edge imbalances
For future research, these methods could be adapted to phylogenetic placement data.
To this end, they need to be extended from abundances ``placed'' on the tips of the OTU tree
to masses placed along the branches of a reference tree.
As balances are a transformation that yields orthogonal components (one for each node of the tree),
issues like the normalization of compositional data do not arise.
Applying these methods to placements instead of OTUs allows for more detailed analyses.
Furthermore, using a fixed reference tree instead of one inferred from the OTUs present in a set of samples
allows comparative studies across datasets.
With samples being represented as a vector of balances,
many standard tools for visualization, ordination, and clustering of data in the euclidean space
could be readily applied to phylogenetic placement data.
Lastly, visualizations similar to our Edge Correlation (\secref{ch:Visualization:sec:Methods:sub:EdgeCorrelation})
could be achieved with such data, by relating the balance per node with meta-data features.
Such a visualization would highlight nodes that exhibit a strong correlation
between changes in the balance of their subtree with environmental variables,
while solving many of the issues of compositional data that our approach might suffer from.

\todo{ maybe add this instead as an outlook idea: } \verb|https://microbiomejournal.biomedcentral.com/articles/10.1186/s40168-018-0584-3|
\todo{from Justin, 11/21/18, 15:49:
One thought: Have you considered building a poisson or multinomial model and letting the PhILR transform be the link function?
We have a new paper using this strategy.  It works quite well. It relies on either a prior (bayesian) or regularization
(frequentist-ish) to ensure that the logs don't push estimates to negative infinity. In some senses it works like model
based zero-imputation. We find it works much better than pseudo-counts. Using this strategy the count model takes
the place of the taxa weights (which are then not needed/used).
}

As we showed, even without the use of such balance approaches,
\emph{unsupervised} methods can be extended to phylogenetic placement data,
and thus are valuable tools for metagenomic data exploration.
It might thus be interesting to also adapt other types of unsupervised machine learning methods to such data,
such as different clustering \cite{Kriegel2011} or dimensionality reduction \cite{VanDerMaaten2008} methods,
or visualization \cite{Laczny2014} and anomaly detection approaches.

The adaptation of \emph{supervised} machine learning approaches to phylogenetic data might however be more complicated.
In recent years, several approaches have been proposed for metagenomic sequence data,
using large-scale machine learning \cite{Vervier2015,Soueidan2015}
and deep learning \cite{Angermueller2016,Fiannaca2018} techniques.
These approaches consider each sequence to be a data point, meaning that enough training data are available.
They are applied for tasks such as OTU clustering, sequence classification, taxonomic assignment, and gene prediction.
In this thesis however, we are interested in whole metagenomic samples (instead of single sequences)
as the data points to be considered.
Current datasets are often not large enough (in number of distinct samples)
to allow for robust learning on this level of granularity without over-fitting the training data \cite{Angermueller2016}.
There are however some recent approaches in this direction, which focus on comparative studies \cite{Soueidan2015}.
They are mainly based on features such as
abundances and presence/absence patters of sequences (or \acsp{OTU}) within the samples \cite{Pasolli2016},
or use so-called $k$-mers (subword occurrences within sequences) as features \cite{Asgari2015}.
These features are simple enough to allow training with current dataset sizes.
However, given the growth rate of metagenomic sequencing,
sufficient training data for more detailed analyses are likely to become available in the future.
Hence, incorporating phylogenetic information in supervised machine learning methods might yield
improved accuracy as well as better interpretability.
This could be used for tasks such as classification of samples,
% (e.\,g., prediction of meta-data features such as health status with respect to a disease)
and prediction and regression with respect to per-sample meta-data features.
Placement-based balances and phylogenetic factorization
are potential ways to approach this idea \cite{Silverman2017,Washburne2017a}.
\todo{check the above!}

Notable supervised approaches that are interesting and promising for problems in the life sciences
are \emph{neural networks}, and \emph{deep learning} in particular \cite{Skansi2018}.
Such methods are in principle well suited for the complex and high-dimensional data produced in these domains.
Hence, first approaches have recently been described
for a variety of use cases in biology and medicine \cite{Mamoshina2016,Min2017,Ching2018}.
There are however some general issues that need to be solved when employing deep learning.
Firstly, similar to other supervised methods, the amount of available (labeled) data in current biological studies
is often too limited for training deep neural networks \cite{Min2017,Ching2018}.
Secondly, deep learning is often considered a ``black box'' that lacks interpretability and testability;
there are however recent approaches to alleviate this \cite{Min2017,Reiman2018,Wainberg2018}.
% how users should interpret these models to make testable hypotheses about the system under study remains an open challenge.
Future data collection and method development efforts are likely able to overcome these obstacles \cite{Wainberg2018}.
Then, deep learning approaches that take phylogenetic information into account
could be expedient tools for analyzing and understanding metagenomic data.

There have also been first attempts that directly use tree structures and phylogenetic trees in neural networks.
For example, the the architecture of a neural network can be modeled
according to a given graph structure \cite{Scarselli2009,Bronstein2017}.
In such models, data on the nodes of the graph is propagated across the graph during the training of the network,
which allows to exploit the additional information encoded in the graph.
This could be used for phylogenetic trees as well, using the tree as a fixed graph,
and the distribution of phylogenetic placements as the data that the system is trained on.
A very recent and promising idea is to embed the evolutionary information of phylogenetic trees into the euclidean space,
and then use convolutional neural networks (CNNs) on the resulting matrices \cite{Fioravanti2018,Reiman2017,Reiman2018}.
These methods are able to accurately predict meta-data features of a set of metagenomic samples,
given the per-sample taxonomic profile.
In detail, these methods infer a tree from the OTUs present in the dataset,
and use the OTU abundances per sample to train the network.
In consequence, the underlying tree differs in each study, which hinders the use of pre-trained networks,
for example in medical applications, where data of new patients needs to be classified.
Instead, one could easily embed a fixed reference tree in a similar manner,
and use phylogenetic placement distributions on the tree instead of abundances for the training.
% These recent methods can be apply with good prospects to metagenomic data, and to phylogenetic placement in particular.

\paragraph{Research Questions}
\label{ch:ConclusionOutlook:par:ResearchQuestions}

% future of phylogeny-aware methods, in particular of phylogenetic placement:
% concepts such as deep learning and machine learning in general
% There is however also potential in the development of more concrete methods for particular applications and use cases.
% Machine learning techniques could for example be used to this end,
% as tools for developing methods for the following research questions
% and provide ample prospects for future development.
% Using such tools and methods can help answer important research questions in biology,
% This can help to answer important research questions

The general concepts and techniques described above %are of current interest in the research community, and
can be employed for developing more concrete methods to help answering novel research questions.
% for example in fields such as biogeography, ecology, and human health.
We here present some ideas for future applications and use cases in metagenomics,
to which phylogeny-aware methods, and phylogenetic placement in particular, could be adapted
by using these techniques.

% a first step to answer known questions in a new phylogeny-aware manner ...
As mentioned in \secref{ch:Introduction:sec:ContributionOverview}, we are currently working on \toolname{scrapp},
which stands for ``Species Counting on Reference trees viA Phylogenetic Placements''.
It is a tool for estimating the species diversity in metagenomic samples per branch of a reference tree,
and hence is useful for discovering and describing novelty in microbial data.
Thus, \toolname{scrapp} reworks a common question in biology (species counting and diversity estimation),
and brings it into a phylogenetic context,
where results can be visualized and interpreted using the additional information of the reference tree.
It is build as a pipeline that combines several tools of our lab, such as
\toolname{EPA-ng} \cite{Barbera2018}, \toolname{ParGenes} \cite{Morel2018}, and \toolname{mPTP} \cite{Kapli2017},
and hence is also a first step in the direction of integrated pipelines as explained above.

% Apart from species diversity, there are
% Estimating species diversity is an important task for metagenomic studies in general.
% and hence in demand by the research community.

A related question is that of estimating the microbial community composition in a metagenomic sample,
for example in the form of per-taxon relative abundances and taxonomic assignments \cite{Lindgreen2016}.
We explained the compositional nature of phylogenetic placements in
\secref{ch:Foundations:sec:PhylogeneticPlacement:sub:PlacementProcessing:par:EdgeMasses},
and described our ad-hoc implementation for taxonomic assignment based on phylogenetic placement in %the \toolname{gappa} tool in
\secref{ch:AutomaticTrees:sec:Evaluation:sub:TaxonomicAssignmentProfiling}.
In order to keep up with methods specifically developed for taxonomic assignment
\cite{Lindgreen2016,Sczyrba2017,Almeida2018},
our implementation could be extended and refined,
and hence become a tool that uses phylogenetic context to answer questions of community composition.

% There are furthermore some specific fields in which we see an opportunity for phylogeny-aware methods,
% for example biogeography, ecology, and medicine. %human health.

Further fields in which we see an opportunity for phylogeny-aware methods are biogeography and ecology,
which ask questions concerning the distribution of species across geographic space and through time,
and their interactions with each other, respectively.

Research has been conducted looking into the biogeographical patterns of microbial communities
\cite{HughesMartiny2006}, and there are also evolutionary approaches \cite{Cox2016}.
This could for instance be extended to phylogenetic placements by combining them with geographical data:
The KR distance between two metagenomic samples is a pairwise measure;
similarly, the geographic locations from which the samples were extracted yield pairwise distances between samples.
These measures could be used to infer microbial patterns that change with geographical distance.
This is similar to correlating the placement distribution with per-sample meta-data
as we described in \secref{ch:Visualization:sec:Methods:sub:EdgeCorrelation},
but requires some other form of correlation measure.
% pairwise measures could be used for this.
We conducted some tests in this direction, but unfortunately,
the available datasets were shown to not exhibit such biogeographical patterns \cite{Lentendu2018}.

Instead of location, time is also be used for comparing metagenomic samples.
In such studies, the same location is sampled at different points in time,
thus allowing to investigate the dynamics, variations, and (potentially) periodic patterns
experienced in the microbial communities across time \cite{Costello2009,Faust2015}.
This is typically assessed using the taxonomic composition of the samples.
By extending such methods to phylogenetic placements,
the time-dependent patterns could be displayed on a phylogenetic tree,
which again might add useful details for interpreting and visualizing them.

% instead of assessing dependencies between species abundances and external factors (such as meta-data features, location, time),
Furthermore, in the field of ecology, the co-occurrence patterns between species
have been researched for many habitats and ecosystems \cite{Lima-Mendez2015,Villalobos2017,Zelezniak2015}.
These interactions are often visualized in so-called co-occurrence networks \cite{Chaffron2010,Faust2012},
which aggregate abundances at a high taxonomic level,
and show links that indicate co-presence and exclusion between the depicted taxa.
Unfortunately, even at high taxonomic levels, these networks are often very complex and hard to interpret,
while potentially omitting valuable details of interactions on lower taxonomic levels.
Thus, it might be interesting to visualize co-occurrences with respect to phylogenetic information.
This entails the challenge to simplify the visualization and only show relevant co-occurrences,
for example by highlighting branches of the reference tree with a high correlation with other branches.
% Such methods might then be

Throughout this thesis, we used the \acf{BV} \cite{Srinivasan2012} and the \acf{HMP} \citep{Huttenhower2012,Methe2012}
datasets as examples for health-related applications of our methods.
Many of the above research questions are also interesting from a medical point of view,
and can be readily applied to other human datasets to assess health status.
In recent years, there have been studies that conduct phylogenetic analyses of widely researched diseases
such as HIV/AIDS \cite{Castro-Nallar2012,Brenner2013} and different types of cancer \cite{Brown2017a,Abbosh2017}.
Thus, we see large potential in phylogeny-aware methods for diagnostics, and human health in general.

In conclusion, the development of novel methods that use phylogenetic information
is highly promising for many fields of research in medicine and the life sciences.
In particular, methods based on phylogenetic placement allow to process large metagenomic datasets,
and to visualize them in different ways using the underlying reference tree as an additional source
of information that aids interpretation of the results.
