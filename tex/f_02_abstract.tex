\selectlanguage{ngerman}
%\begin{abstract}
\pdfbookmark{Zusammenfassung}{Zusammenfassung}
\section*{Zusammenfassung}
\vspace*{1em}

Die DNS (englisch: DNA) bildet die vererbbare Grundlage allen bekannten Lebens auf dem Planeten.
Entsprechend wichtig ist ihre ``Entschl\"usselung'' f\"ur die Biologie im Allgemeinen,
und f\"ur die Erforschung der evolution\"aren Zusammenh\"ange verschiedener biologischer Artern im Besonderen.
In den letzten Jahrzehnten hat eine rasante technologische Entwicklung im Bereich der DNS-Sequenzierung stattgefunden,
die auch auf absehbare Zeit noch nicht zum Stillstand kommen wird.
Die biologische Forschung hat daher den Bedarf an computer-gest\"utzten Methoden erkannt,
sowohl in Bezug auf die Speicherung und Verarbeitung der immensen Datenmengen, die bei der Sequenzierung anfallen,
als auch in Bezug auf deren Analyse und Visualisierung.

Eine grundlegene Fragestellung ist dabei die nach dem Stammbaum des Lebens,
der die evolution\"are Verwandtschaft der Arten beschreibt.
Diese Wissenschaft wird Phylogenetik, und die resultierenden Strukturen phylogenetische B\"aume genannt.
H\"aufig basieren diese B\"aume auf dem Vergleich von DNS-Sequenzen der Arten,
mit der Idee, dass Arten mit \"ahnlicher DNS auch im Baum nah beieinander liegen.
Die Berechnung eines solchen Baumes aus DNS-Daten kann als Optimierungsproblem formuliert werden,
und ist durch die stetig wachsende Menge an Daten daher f\"ur die Informatik eine Herausforderung.
% Gerade f\"ur die Mikrobiologie, die sich unter anderem mit
Aktuell besch\"aftigt sich die Mikrobiologie zum Beispiel mit der Erkundung und Erforschung von Proben (Samples),
die aus Wasser, dem Erdreich, und \"ahnlichen Umgebungen gewonnen wurden:
Welche mikrobischen Arten, Bakterien und andere Einzeller, bewohnen diese Proben?
Das zugeh\"orige Forschungsfeld ist die Meta-Genetik.
Einen verl\"asslichen Stammbaum f\"ur die aber-millionen an Sequenzen aus solchen Proben zu errechnen
ist praktisch unm\"oglich.
Eine Alternative bietet die phylogenetische Platzierung der Sequenzen auf einem gegebenen Referenz-Baum von bekannten Arten
(so genanntes phylogenetisches Placement):
Hierbei wird ein Baum aus Referenz-Sequenzen bekannter Arten gew\"ahlt,
der m\"oglichst viel der in den Proben zu erwartenden Artenvielfalt abdeckt,
und dann f\"ur jede Sequenz aus den Proben die n\"achste Verwandtschaft innerhalb des Baumes bestimmt.
Dies resultiert in einer Zuordnung von Sequenzen auf die Positionen verwandter Arten im Referenz-Baum.
Diese Zuordnung kann auch als Verteilung der Sequenzen auf dem Baum verstanden werden:
In dieser Interpretation kann man beispielsweise erkennen,
welche Arten (und deren Verwandtschaft) besonders h\"aufig in den Proben vertreten sind.

Diese Arbeit besch\"aftigt sich mit neuen Methoden zur Vor- und Nachbereitung, Analyse, und Visualisierung
rund um den Kernbereich des phylogenetischen Placements von DNS-Sequenzen.
Zun\"achst stellen wir eine Methode vor, die einen geeigneten Referenz-Baum f\"ur die Platzierung liefern kann.
Die Methode hei\ss{}t \emph{PhAT} (Phylogenetic Automatic (Reference) Trees),
und nutzt Datenbanken bekannter DNS-Sequenzen, um geeigenete Referenz-Sequenzen f\"ur den Baum zu bestimmen.
Die durch PhAT produzierten B\"aume sind beispielsweise dann interessant,
wenn die in den Proben zu erwartende Artenvielfalt noch nicht bekannt ist:
In diesem Fall kann ein breiter Baum, der viele der bekannten Arten abdeckt, helfen, neue, unbekannte Arten zu entdecken.
Im gleichen Kapitel stellen wir au\ss{}erdem zwei Behilfs-Methoden vor,
um den Prozess und die Berechnungen der Placements von gro\ss{}en Datens\"atzen zu beschleunigen und zu erm\"oglichen.
Zum einen stellen wir Multilevel-Placement vor,
mit dem besonders gro\ss{}e Referenz-B\"aume in kleinere, geschachtelte B\"aume aufgeteilt werden k\"onnen,
um so schnellere und detalliertere Platzierungen vornehmen k\"onnen, als auf einem einzelnen gro\ss{}en Baum m\"oglich w\"aren.
Zum anderen beschreiben wir eine Pipeline, die durch geschickte Lastverteilung und Vermeidung von Duplikaten
den Prozess weiter beschleunigen kann.
Dies eignet sich insbesondere f\"ur gro\ss{}e Datens\"atze von zu platzierenden Sequenzen,
und hat die Berechnungen erst erm\"oglicht, die wir zum testen der im weiteren vorgestellten Methoden ben\"otigt haben.

Im Anschluss stellen wir zwei Methoden vor, um die Placement-Ergebnisse verschiedener Proben miteinander zu vergleichen.
Die Methoden, \emph{Edge Dispersion} und \emph{Edge Correlation}, visualisieren den Referenz-Baum derart,
dass die in Bezug auf die Proben interessanten und relevanten Regionen des Baumes sichtbar werden.
Edge Dispersion zeigt dabei Regionen, in denen sich die H\"aufigkeit der in den Proben vorhandenen mikrobischen Arten
besonders stark zwischen den einzelnen Proben unterscheided.
Dies kann als erste Erkundung von neuen Datens\"atzen dienen,
und gibt Aufschluss \"uber die Varianz der H\"aufigkeit bestimmter Arten.
Edge Correlation hingegen bezieht zus\"atzlich Meta-Daten mit ein, die zu den Proben gesammelt wurden.
Dadurch k\"onnen beispielsweise Abh\"angigkeiten zwischen H\"aufigkeiten von Arten und Faktoren wie dem pH-Wert des Bodens
oder dem Nitrat-Gehalt des Wassers, aus dem die Proben stammen, aufgezeigt werden.
Es hat damit \"Ahnlichkeiten zu einer bestehenden Methode names Edge PCA,
die ebenfalls relevante Regionen des Baumen identifizieren kann,
allerdings die vorhandenen Meta-Daten nur indirekt einbeziehen kann.

Eine weitere Fragestellung ist die Gruppierung (Clustering) von Proben anhand von Gemeinsamkeiten,
wie beispielweise einer \"ahnlichen Verteilungen der Sequenzen auf dem Referenz-Baum.
Anhand geeigneter Distanz-Ma\ss{}e wie der Kantorovich-Rubinstein-Distanz (KR-Distanz)
k\"onnen \"Ahnlichkeiten zwischen Proben quantifiziert werden, und somit ein Clustering erstellt werden.
F\"ur gro\ss{}e Datens\"atze mit hunderten und tausenden von einzlnen Proben sto\ss{}en bestehende Methoden
f\"ur diesen Einsatzzweck, wie zum Beispiel das so genannte Squash Clustering, an ihre Grenzen.
Wir haben daher die $k$-means-Methode derart erweitert, dass sie f\"ur Placement-Daten genutzt werden kann.
Dazu pr\"asentieren wir zwei Methoden, \emph{Phylogenetic $k$-means} und \emph{Imbalance $k$-means},
die verschiedene Distanzma\ss{}e (KR-Distanz, und ein weiteres geeignetes Ma\ss{}) nutzen,
um B\"aume mit \"ahnlichen Verteilungen von platzierten Sequenzen zu gruppieren.
Sie betrachten jede Probe als einen Datenpunkt,
und nutzen die zugrunde liegende Struktur des Referenz-Baumes f\"ur die Berechnungen.
Mit dieser Methode k\"onnen auch Datens\"atze mit zehntausenden Proben verarbeitet werden,
und Clusterings und \"Ahnlichkeiten von Proben erkannt und visualisiert werden.

Wir haben au\ss{}erdem ein Konzept namens \emph{Balances} f\"ur Placement-Daten adaptiert,
welches urspr\"unglich f\"ur so genannte OTU-Sequenzen (Operational Taxonomic Units) entwickelt wurde.
Balances erlauben eine Beschreibung des Referenz-Baumes und der darauf platzierten Sequenzen,
die ganze Gruppen von Referenz-Arten zusammenfasst,
statt jede Art einzeln in die Berechnungen einflie\ss{}en zu lassen.
Diese Beschreibung der Daten bietet verschiedene Vorteile f\"ur die darauf basierenden Analysen,
wie zum Beispiel eine Robustheit gegen\"uber der exakten Wahl der Referenz-Sequenzen,
und einer anschaulichen Beschreibung und Visualisierung der Ergebnisse.
Insbesondere aus mathematischer Sicht sind Balances f\"ur die Analyse interessant,
da sie problematische Artefakte aufgrund der kompositionellen Natur meta-genetischer Daten beheben.
Im Zuge dieser Arbeit dienen Balances haupts\"achlich als Zwischenschritt zur Daten-Repr\"asentation.

Eine Anwendung von Balances ist die so genannte \emph{Phylo-Factorization}.
Diese recht neue Methode teilt einen gegebenen Baum derart in Sub-B\"aume ein,
dass jeder Sub-Baum eine Gruppe von Arten darstellt, die in Bezug auf gegebene Meta-Daten pro Probe eine Rolle spielt.
Dadurch k\"onnen beispielsweise Gruppen identifiziert werden,
deren evolution\"are Merkmale sich in Abh\"angigkeit von Meta-Daten wie pH-Wert angepasst haben im Vergleich zu anderen Gruppen.
Dies ist \"ahnlich zur oben genannten Edge Correlation, aber kann zum einen durch geschickte mathematische Ans\"atze
(insbesondere der Nutzung von Generalized Linear Models)
mehrere Meta-Daten gleichzeitig einbeziehen, und zum anderen auch verschachtelte Gruppen finden.
Die zugrunde liegenden Ideen dieser Methoden bieten einen gro\ss{}en Spielraum sowohl f\"ur Analysen von Daten,
als auch f\"ur Weiterentwicklungen und Erg\"anzungen f\"ur verwandte Fragestellungen.
Wir haben diese Methode f\"ur Placement-Daten adaptiert und erweitert,
und stellen diese Variante, genannt \emph{Placement-Factorization}, vor.
Im Zuge dieser Adaption haben wir au\ss{}erdem verschiedene erg\"anzende Berechnungen und Visalisierungen entwickelt,
die auch f\"ur die urspr\"ungliche Phylo-Factorization n\"utzlich sind.

Alle genannten neuen Methoden wurden ausf\"uhrlich getestet
in Bezug auf ihre Eignung zur Erforschung von mikrobiologischen Zusammenh\"angen.
Wir haben dazu verschiedene bekannte Datzens\"atze von DNS-Sequenzen aus Wasser- und Bodenproben,
sowie Proben des menschlichen Mikrobioms, verwendet und
diese auf geeigneten Referenz-B\"aumen platziert.
Anhand dieser Daten haben wir zum einen die Plausibilit\"at der durch unsere Analysen erzielten Ergebnisse gepr\"uft,
als auch Vergleiche der Ergebnisse mit \"ahnlichen, etablierten Methoden vorgenommen.
S\"amtliche Analysen, Visualisierungen, und Vergleiche werden in den jeweils entsprechenden Kapiteln
vorgestellt, und die Ergebnisse dargestellt.
Alle Tests zeigen, dass unsere Methoden auf den getesteten Datens\"atzen zu Resultaten f\"uhren,
die konsistent mit anderen Analysen sind, und geeignet sind, um neue biologische Erkenntnisse zu gewinnen.

S\"amtliche hier vorgestellten Methoden sind implementiert in unserer Software-Biblio\-thek \toolname{genesis},
die wir im Zuge dieser Arbeit entwickelt haben.
Die Bibliothek ist in modernem \texttt{C++11} geschrieben, hat einen modularen und funktions-orientierten Aufbau,
ist auf Speichernutzung und Rechengeschwindigkeit optimiert, und nutzt vorhandene Multi-Prozessor-Umgebungen.
Sie eignet sich daher sowohl f\"ur schnelle Tests von Prototypen,
als auch zur Entwicklung von Analyse-Software f\"ur Endanwender.
Wir haben \toolname{genesis} bereits erfolgreich in vielen unserer Projekte eingesetzt.
Insbesondere bieten wir s\"amtliche hier pr\"asentierten Methoden \"uber unser Software-Tool \toolname{gappa} an,
das intern auf \toolname{genesis} basiert.
Das Tool stellt einen einfachen Kommandozeilen-Zugriff auf die vorhandenen Analysemethoden bereit,
und bietet ausreichend Optionen f\"ur die Analysen der meisten End-Anwender.

Im abschlie\ss{}enden Kapitel wagen wir einen Ausblick in weitere Forschungsm\"oglichkeiten im Bereich der
Methoden-Entwicklung f\"ur Meta-Genetik im Allgemeinen, und der placement-basierten Methoden im Speziellen.
Wir benennen verschiedene Herausforderungen in Bezug auf die Nutzbarkeit solcher Methoden f\"ur Anwender
und ihrer Skalierbarkeit f\"ur immer gr\"o\ss{}er werdende Datens\"atze.
Au\ss{}erdem schlagen wir verschiedene weitergehende Ans\"atze vor,
die zum Beispiel auf neuronalen Netzwerken und Deep Learning basieren k\"onnten.
Mit aktuellen Datens\"atzen w\"aren solche Methoden nicht robust trainierbar;
durch das in Zukuft zu erwartenden Wachstum an Daten kann dies allerdings bald in den Bereich des M\"oglichen kommen.
Schlie\ss{}lich identifizierenden wir einige tiefer gehende Forschungsfragen aus der Biologie und Medizin,
bei deren Beantwortung unsere Methoden in Zukunft helfen k\"onnen.

%\end{abstract}
\blankpage

\selectlanguage{english}
%\begin{abstract}
\pdfbookmark{Abstract}{Abstract}
\section*{Abstract}
\vspace*{1em}

The DNA is the hereditary basis of all known life on the planet.
Deciphering this ``code of life'' is hence of key importance for biology in general,
and for unravelling the evolutionary relationships between biological species in particular.
The last few decades have seen rapid technological advances in DNA sequencing,
with no slowdown of this trend being in sight.
Research in biology hence has high demand for computational methods,
both with respect to storage and processing of these huge datasets,
and to analysis and visualization thereof.

A basic question is that of the tree of life, which describes the evolutionary relationship between species.
This field of science is called phylogenetics, and resulting structures are called phylogenetic trees.
Often, these trees are based on the comparison of DNA sequences of the species,
and are build on the idea that species with similar sequences are located on nearby branches of the tree.
The inferrence of such a tree based on DNA data can be formulated as an optimization problem,
and poses a challange for computer science due to the ever increasing amount of available data.
For example, a current directive in micro-biology is to investigate the composition of samples
taken from environments such as water or soil:
Which microbial species, bacteria and other single cellular organisms, are present in these samples?
This field of research is called meta-genomics.
It is infeasible to compute a robust phylogenetic tree for the millions of sequences obtained from such samples.
An alternative approach is the so called phylogenetic placement of the sequences on a given reference tree
of known species:
Given a tree of reference sequences of known species that covers the expected diversity in the samples as much as possible,
the evolutionary relation of the sequences in the samples to the reference tree is determined.
This yields a mapping from sequences to positions of related species in the reference tree.
This mapping can also be understood as a distribution of sequences on the tree:
This interpretation allows for example to visualize
which species (and their next of kin) are frequently present in the samples.

In this work, we developed novel methods for pre- and post-processing, analysis, and visualization
of phylogenetic placement of DNA sequences.
Firstly, we present a method to automatically obtain a suitable reference tree to be used for placement.
The method is called \emph{PhAT} (Phylogenetic Automatic (Reference) Trees),
and uses databases of known DNA sequences in order to determine suitable reference sequences.
The trees produced by PhAT are for example useful when the expected species diversity in the samples is not yet known:
In this case, a broad tree that covers many of the known species, can help to discover novel, unknown species.
In the same chapter, we also present two auxiliary methods that accelerate and enable the process and the computations
needed for the placement of very large datasets.
On the one hand, we present Multilevel-Placement, that uses a divide-and-conquer approach to split large reference trees
into small, nested trees.
It thereby improves speed and accuracy of the placement process compared to using one large reference tree.
On the other hand, we describe a pipeline that maximizes load distribution and further accelerates the placement process
by avoiding duplicate computations.
This is particularly suited for large datasets, and was a necessary improvement to enable the computations needed
for the tests of the further methods presented in this work.

Subsequently, we present two methods to compare the placement results of distinct samples with each other.
The methods, \emph{Edge Dispersion} and \emph{Edge Correlation}, visualize the reference tree so that
the interesting and relevant regions of the tree (with respect to the samples) become apparent.
Edge Dispersion shows regions where the frequency of microbial species in the samples differs most in between samples.
This can serve as a first exploration of a dataset, and indicates the variance of the occurrences of species.
Edge Correlation on the other hand additionally takes meta-data into account that was collected per sample.
It hence can for example show the dependencies between occurrences of species and environmental factors such as the
pH-value of the soil, or the nitrate content of the water where each sample was taken from.
This bears some similarity to an existing method called Edge PCA,
which also highlights relevant regions of the reference tree,
but can only indirectly incorporate meta-data features.

Another research question is that of grouping or clustering of samples based on similarities,
for example a similar distribution of sequences on the reference tree.
By using suitable distance measures such as the Kantorovich-Rubinstein distance (KR distance),
similarities between samples can be quantized, and leveraged to cluster them.
For large datasets with hundreds to thousands of distinct samples,
existing methods for this purpose, such as the so called Squash Clustering, reach their scalability limits.
We thus extended the $k$-means method to be applicable to placement data.
To this end, we present two methods, \emph{Phylogenetic $k$-means} and \emph{Imbalance $k$-means},
that use two different distance measures (KR distance, and another suitable measure)
to cluster trees with similar distributions of placed sequences.
These methods regard each sample as a distinct data item,
and use the underlying structure of the reference tree for the computations.
This method can be applied to datasets with tens of thousands of samples,
in order to find clusters and similarities between samples, and visualize these.

Furthermore, we adapted a concept called \emph{Balances} to placement data,
which was originally intended for so called OTU sequences (Operational Taxonomic Units).
Balances allow for a description of the reference tree and the sequences placed on it
in a way that summarizes groups of referece species, instead of taking each species into account individually.
This description of the data offers several advantages for subsequent analysis steps;
for example, it is robust in terms of the exact choice of reference sequences,
and offers an intuitive way of visualizing results obtained from these analyses.
Balances are in particular helpful from a mathematical standpoint,
as they circumvent problematic artifacts due to the compositional nature of meta-genomic data.
In this work, balances are mainly used as an intermediate step for data representation purposes.

One application of balances is the so called \emph{Phylo-Factorization}.
This relatively recent method splits a given tree into a set of sub-trees
so that each sub-tree represents a group of species that are relevant with respect to the meta-data features
of the given samples.
This allows for example to identify groups whose evolutionary traits changed depending on meta-data such as pH-value
in comparison to other groups.
This is similar to the Edge Correlation method mentioned above,
but further allows to incorporate several meta-data features at once and can find nested groups of species,
by leveraging mathematical approaches such as Generalized Linear Models.
The underlying concepts of Phylo-Factorization are versatile both for data analysis
as well as for extension and adaptation to releated research questions.
We have adapted the method to placement data, and present this variant, which we call \emph{Placement-Factorization}.
Additionally, we developed several auxiliary computations and visualizations of the results
that are also useful for the original Phylo-Factorization.

All mentioned novel methods were extensively tested with respect to their suitability for discovering biological knowledge.
To this end, we used several DNA sequence datasets from water and soil, as well as from the human body,
and phylogenetically placed them on suitable reference trees.
Based on this, we tested the plausibility of the results obtained from our analyses,
and compared them to the results of simiar, established methods.
All analyses, visualizations, and comparisons are described in detail in the respective chapters,
along with the results and their interpretations.
All tests show that our methods yield results on the datasets that are consistent with other types of analyses,
and are suitable for discovering novel biological knowledge.

The methods presented here are implemented in our software library \toolname{genesis},
which we devloped alongside this work.
The library is written in modern \texttt{C++11}, has a modular and function-oriented design,
is optimized for memory consumption and computing speed, and leverages multi-core environments.
It is hence suitable for rapid testing of prototype software, as well as for developing analysis software for end users.
We already have successfully deployed \toolname{genesis} in several of our projects.
In particular, all presented methods are incorporated into our command line tool \toolname{gappa},
which is internally based on \toolname{genesis}.
The tool has a simple command line interface to our analysis methods that offers sufficient options for most end users.

In the final chapter, we dare an outlook into possible research directions for method development in meta-genetics
in general, and placement-based methods in particular.
We identify several challenges with respect to the usability of such methods for researchers,
and their scalability to ever larger datasets.
Furthermore, we suggest several further approaches, for instance based on neural networks and deep learning.
With current datasets, such methods cannot robustly be trained;
due to the expected growth of data in the near future however, such approaches are likely to become feasible.
Finally, we identify some in-depth research questions from the fields of biology and medicine
for which our methods might be useful in the future.


% metagenomics, data, phylo placement

% First, we present an approach to automatically obtain reference phylogenetic trees
% called \emph{\acfp{PhAT}} from large reference sequence databases.
% We show that the \acp{PhAT} are valuable and accurate reference trees
% for conducting phylogenetic placement of metagenomic sequences.
% % by measuring the placement accuracy they yield, and by evaluating their properties when placing empirical datasets.
% They can also be used for taxonomic assignment of sequences,
% or be tailored to specific clades of the underlying reference database.
% We moreover present approaches and pipelines to enable and accelerate phylogenetic placement of large datasets.
% Our \emph{multi-level} placement approach can be used for large, diverse datasets
% where a single reference tree for all species of interest is too large for analysis and interpretation,
% without sacrificing placement accuracy.
% Our pre-processing pipeline furthermore helps to conduct large-scale analyses
% of hundreds to thousands of environmental sequence samples.
%
% Next, we described methods for visualizing such datasets in \chpref{ch:Visualization}.
% The methods allow to detect differences between samples (\emph{Edge Dispersion}),
% as well as correlations with per-sample meta-data (\emph{Edge Correlation}),
% and thus are intended for similar use cases as the established Edge PCA \cite{Matsen2011a}.
% However, our novel methods directly visualize important features of the samples (an their meta-data)
% on the underlying reference tree, which allows interpretation in a phylogenetic context.
%
% all methods are evaluated in terms of their scaling and speed
% as well as
% using empirical datasets to show that they are able to reproduce known
% things and yield novel interpretations

%\end{abstract}
