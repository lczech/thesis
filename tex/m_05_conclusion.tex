% ######################################################################################################################
%         Conclusion and Outlook
% ######################################################################################################################

\chapter{Conclusion and Outlook}
\label{ch:ConclusionOutlook}

In this thesis, we made several contributions to the field of computational phylogenetics,
and phylogenetic placement in particular.
The methods were already described in two peer-reviewed publications \cite{Czech2018,Czech2018a},
and the respective data and scripts were made available at \url{http://github.com/lczech/placement-methods-paper}.
The implementation is available in our \toolname{genesis} library (\url{https://github.com/lczech/genesis});
we furthermore offer a ready-to-use command line interface for the presented methods
via our \toolname{gappa} tool (\url{https://github.com/lczech/gappa}).
This chapter briefly summarizes our contributions and gives an outlook for future development and research questions
in the field of phylogenetic placement.

Being first introduced between 2009 and 2010 \cite{Berger2009,Matsen2010a},
phylogenetic placement and its downstream analysis methods are still in their infancy,
but are becoming more and more popular in the research community.
Their advantages include greater interpretability and visualizability of results,
at the cost of additional methodological and computational complexity.
There are hence still obstacles to overcome before phylogeny-aware methods (such as phylogenetic placement)
can reach maturity and widespread use in the research community:
% generally less prevalent than simpler analysis methods,
% such as those based on sequence similarity.

\begin{itemize}
    \item \textbf{Usability:}
          Current software pipelines for phylogenetic placement are mostly build from separate programs and tools
          that are not well integrated with each other, and only bound together via intermediate files and scripts.
          This induces costs for first-time users of such tools, such as the manual compilation and setup of the software,
          as well as costs for ad-hoc scripting solutions to connect different steps of the pipeline.
    \item \textbf{Scalability:}
          The amount of sequence data is growing exponentially (see \secref{ch:Introduction:sec:Motivation}),
          currently doubling roughly every seven months \cite{Stephens2015}.
          Phylogeny-aware analysis methods for sequence data are generally more compute-intense than, e.\,g.,
          methods based on sequence similarity.
          The speed and scalability of these tools thus
          need to stay on a par with the amount of sequencing data being produced.
    \item \textbf{Analysis Methods and Research Questions:}
          Downstream analysis methods that take phylogenetic information into account
          are not yet as versatile and matured as classical data analysis methods.
          Hence, researchers might refrain from using phylogenetic placement and resort to established methodology.
          There are however many research questions that might benefit from analyses using phylogeny-aware methods,
          for example in disciplines such as bio-geography or health and medicine.
%           Methods are missing maybe because of the complexity.
\end{itemize}

We discuss our contributions to the field with respect to these obstacles in the following,
and describe potential research directions for overcoming them.

\paragraph{Usability and Scalability}
\label{ch:ConclusionOutlook:par:UsabilityScalability}

In \chpref{ch:AutomaticTrees} we presented an approach to automatically obtain
reference trees for phylogenetic placement called \emph{\acfp{PhAT}} using large reference sequence databases.
We showed that \acp{PhAT} are valuable and accurate trees for
conducting phylogenetic placement and taxonomic assignment of metagenomic sequences.
They thus alleviate the labor-intense and potentially error-prone work of collecting suitable reference sequences,
and might even replace such ``manual'' trees in some applications.
% Hence, an interesting future direction might be to extend and evaluate them for other purposes and use cases such as ???
In the same chapter, we presented a multi-level placement approach as well as our pre-processing pipeline
to enable and speed up phylogenetic placement
of large, diverse datasets with hundreds to thousands of environmental sequence samples.
All these methods simplify and accelerate the processes for researchers
who want to apply phylogenetic placement and respective downstream analyses.

In order to further facilitate to get acquainted with phylogenetic placement without the need for setup and scripting,
online services for quickly testing standard methods might be worth to offer.
For example, uploaded query sequences could be placed on our four \acp{PhAT} used in the evaluation
(see \secref{ch:AutomaticTrees:sec:Evaluation:sub:ReferenceTreeSetup:par:SequenceSelection}),
or on custom reference trees.
These services could further offer standard analysis methods such as Edge~PCA and Squash Clustering
(\secref{ch:Foundations:sec:PhylogeneticPlacement:sub:ExistingMethods}),
or the methods that we presented in this thesis (see Chapters \ref{ch:Visualization} and \ref{ch:Clustering}).

For more advanced or large-scale studies, a custom setup is however usually inevitable,
both in order to scale up the processing using computer clusters as well as to allow customization of the workflow.
% State-of-the-art implementations of phylogenetic placement, such as our \toolname{EPA-ng} \cite{Barbera2018},
% are able to place billions of sequences withing a few hours on computer clusters.
% There are however other steps in typical pipelines that do not yet scale and hence form a bottleneck.
% For example, \toolname{PaPaRa} \cite{Berger2011a,Berger2012},
% a phylogeny-aware tool for aligning query sequences against a reference alignment,
% might benefit from a re-implementation.
Such studies might benefit from offering well-defined integrated pipelines for the basic steps of phylogenetic placement,
with a simple setup for all common functionality,
for instance by using platforms such as \toolname{Galaxy} \cite{Afgan2018}.
Tools and pipelines that integrate different steps of the analysis can furthermore be implemented more efficiently,
as they can make better use of computing infrastructure such as \texttt{\acs{MPI}},
and distribute computations with less overhead for intermediate files and bookkeeping between pipeline steps.

% can be implemented in integrated ways for computer clusters.
% which will be necessary because of the large amounts of data that are coming our way

Moreover, to achieve a tighter interaction of the tools within a pipeline,
and to incentivize experimentation with existing methods and development of novel methods,
file formats need to be flexible and extensible.
For example, the \fileformat{jplace} file standard for phylogenetic placement \cite{Matsen2012}
is based on the \fileformat{JSON} format \cite{JsonMemo,JsonStandard}, and hence easily extensible.
The standard however currently (as of \fileformat{jplace} version 3) lacks support for multiple samples per file,
and additional per-sample or even per-query annotations and other meta-data.
A similar issue applies to file formats for phylogenetic trees, as we showed in \cite{Czech2017}.
In particular, the \fileformat{Newick} format is often extended ad-hoc for specific needs
(this is even done within the \fileformat{jplace} format itself),
with no standardized support from downstream tools or tree viewers.
Proper standards could thus also help to avoid ambiguities and misinterpretations of results.

% before researchers actually tune existing methods to their particular needs
% and start extending and experimenting with existing approaches.
% hinders rapid testing of methods,

With respect to the methods and tools presented in this thesis,
we plan to extend our \toolname{gappa} tool to include more existing analysis methods
and additional pre- and post-processing functions such as filtering, merging and manipulating placement files.
In the long term, we also want to re-implement all functionality offered by the
\toolname{guppy} tool from the \toolname{pplacer} suite \cite{Matsen2010},
and thereby make \toolname{gappa} a more efficient and scalable replacement for \toolname{guppy}.

Furthermore, we are planning to extend the \toolname{genesis} library, which is written in \texttt{C++},
to offer \acs{API} bindings for \texttt{Python} as well.
Hence, \toolname{genesis} could combine the advantages of both programming languages:
For standard functions such as file processing and most internal computations,
it can use our highly efficient \texttt{C++} implementations,
which often make use of multi-threading via \texttt{OpenMP} (Open Multi-Processing) for multi-core computers.
The modular and clear \acs{API} of \toolname{genesis} further allows its use in production code.
At the same time, ad-hoc solutions for pipeline tasks such as file conversions or extraction of certain data
could be implemented as \texttt{Python} scripts.
This also facilitates to use \toolname{genesis} for experimentation and rapid prototyping of novel ideas and methods.

\paragraph{Analysis Methods}
\label{ch:ConclusionOutlook:par:AnalysisMethods}

% downstream methods for phylogenetic placement are not yes as plenty as standard sequence based methods

In \chpref{ch:Visualization}, we described methods for visualizing the phylogenetic placement of large metagenomic datasets.
The methods allow to detect differences between samples (\emph{Edge Dispersion}),
as well as correlations with per-sample meta-data (\emph{Edge Correlation}),
and thus are intended for similar use cases as the established Edge PCA \citep{Matsen2011a}.
However, our novel methods directly visualize important features of the samples (an their meta-data)
on the underlying reference tree, which allows interpretation in a phylogenetic context.
Furthermore, in \chpref{ch:Clustering}, we introduced clustering methods for metagenomic samples
(\emph{Phylogenetic $k$-means} and \emph{Imbalance $k$-means}),
which serve a similar purpose as Squash Clustering \citep{Matsen2011a},
but are better suited for larger datasets.
% The presented methods can also be combined, allowing

These novel methods extend the portfolio of available analysis tools for phylogenetic placement.
There are however still many approaches from the fields of machine learning and data mining
that could be useful to adapt to phylogenetic placement data.
It might further be worth to develop integrative methods that can incorporate heterogeneous features
such as phylogenetic data and different types of meta-data in a combined analysis \cite{Mariette2018}.
% e.\,g., using unsupervised machine learning methods \cite{Mariette2018}.
The challenge for adapting existing methods usually consists in making the methods phylogeny-aware,
for example by having them operate on distributions on trees
(including respective distance measures such as the KR distance) rather than on $\mathbb{R}^d$ vectors.

As we showed, \emph{unsupervised} methods can be extended to phylogenetic placement data,
and thus are valuable tools for metagenomic data exploration.
It might thus be interesting to also adapt other types of unsupervised machine learning methods to such data,
such as different clustering \cite{Kriegel2011} or dimensionality reduction \cite{VanDerMaaten2008} methods,
or visualization and anomaly detection approaches.

The adaptation of \emph{supervised} machine learning approaches to phylogenetic data might however be more complicated.
In recent years, several approaches have been proposed for metagenomic sequence data,
using large-scale machine learning \cite{Vervier2015,Soueidan2015}
and deep learning \cite{Angermueller2016,Fiannaca2018} techniques.
These approaches consider each sequence to be a data point, meaning that enough training data are available.
They are applied for tasks such as OTU clustering, sequence classification, taxonomic assignment, and gene prediction.
In this thesis however, we are interested in whole metagenomic samples (instead of single sequences)
as the data points to be considered.
Current datasets are often not large enough (in number of distinct samples)
to allow for robust learning on this level of granularity.
Recent approaches in this direction focus on comparative studies \cite{Soueidan2015}.
They are mainly based on features such as
abundances and presence/absence patters of sequences (or \acsp{OTU}) within the samples \cite{Pasolli2016},
or use so-called $k$-mers (subword occurrences within sequences) as features \cite{Asgari2015}.
These features are simple enough to allow training with current dataset sizes.
However, given the growth rate of metagenomic sequencing,
sufficient training data for more detailed analyses are likely to become available in the future.
Hence, incorporating phylogenetic information in supervised machine learning methods might yield improvements,
and could be used for tasks such as classification of samples and regression with respect to per-sample meta-data features.

Another interesting and promising research direction is the application of deep learning techniques
to graph structures \cite{Scarselli2009,Bronstein2017}.
In such models, information is propagated across the graph,
which could model placement distributions by interpreting the underlying phylogeny as a graph structure
\todo{}

This however also requires datasets with enough samples for training such models.

very promising: very recently, methods have been suggested

Phylogenetic convolutional neural networks in metagenomics
\cite{Fioravanti2018}

PopPhy-CNN: A Phylogenetic Tree Embedded Architecture for Convolution Neural Networks for
Metagenomic Data
\cite{Reiman2018}

\paragraph{Research Questions}
\label{ch:ConclusionOutlook:par:ResearchQuestions}

% future of phylogeny-aware methods, in particular of phylogenetic placement:
% concepts such as deep learning and machine learning in general
% There is however also potential in the development of more concrete methods for particular applications and use cases.
% Machine learning techniques could for example be used to this end,
% as tools for developing methods for the following research questions
% and provide ample prospects for future development.

The general concepts and techniques described above %are of current interest in the research community, and
can be employed as tools for developing more concrete methods for particular

to help answering particular research questions

we here describe some ideas for future
applications and use cases,
to whic pp could be adapted

% Using such tools and methods can help answer important research questions in biology,
% This can help to answer important research questions
for example
in biology, as well as in subfields such as biogeography, ecology, medicine and human health.
Here, we present some ideas for future research questions to which phylogeny-aware methods,
and phylogenetic placement in particular, could be adapted.


%-----------------------
% Species Diversity Estimation
%-----------------------

\noindent\rule{\textwidth}{0.4pt}

As mentioned in \secref{ch:Introduction:sec:ContributionOverview}, we are currently working on \toolname{scrapp},
which stands for ``Species Counting on Reference trees viA Phylogenetic Placements''.
It is a tool for estimating the species diversity in metagenomic samples per branch of a reference tree,
and hence is useful for discovering novelty in microbial data.

this is important and useful, and is a tool in demand by the research community.
and an example of how to re-work a problem (species counting) into a phylogenetic context and display stuff on the tree

It is build as a pipeline that combines several tools of our lab, such as
\toolname{EPA-ng} \cite{Barbera2018}, \toolname{ParGenes} \cite{Morel2018}, and \toolname{mPTP} \cite{Kapli2017},
and hence is also a first step in the direction of integrated pipelines as explained above.



% Apart from species diversity, there are
Estimating species diversity is an important task for metagenomic studies in general.
% and hence in demand by the research community.


%-----------------------
% Taxonomic Assignment
%-----------------------

\noindent\rule{\textwidth}{0.4pt}

metagenomic studies need to estimate the microbial community composition, for example in the form of per-taxon relative abundances \cite{Lindgreen2016}
taxonomic assignment

our ad-hoc implementation in gappa assign:
a tool that can keep up with methods specifically developed for this purpose
\cite{Lindgreen2016,Sczyrba2017,Almeida2018}
see \secref{ch:AutomaticTrees:sec:Evaluation:sub:TaxonomicAssignmentProfiling}

could be extended to a full tool for tax assignment



There are furthermore some more specific fields in which we see an opportunity for phylogeny-aware methods,
for example biogeography, ecology, and human health.




wiki:
``Biogeography is the study of the distribution of species and ecosystems in geographic space and through geological time.
Organisms and biological communities often vary in a regular fashion along geographic gradients of latitude, elevation, isolation and habitat area.''




%-----------------------
% Biogeography and Ecology
%-----------------------

\noindent\rule{\textwidth}{0.4pt}

pairwise geographic distances between the locations form which environmental samples were extracted

use in combination with pairwise comparisons between the phylogenetic data contained in the sample, for instance, via the KR distance


biogeography stuff (Guillaume Lentendu)

There is reasearch on microbial biogeography:
Microbial biogeography \cite{HughesMartiny2006}
``A large body of research supports the idea that free-living microbial taxa exhibit biogeographic patterns.''

and it also used evolutionary methods:
biogeography evolutionary approach
\cite{Cox2016}

Bacterial Community Variation in Human Body Habitats Across Space and Time \cite{Costello2009}

time series, same location (i.e., forests and lakes, or individual humans)
periodic patterns:

Metagenomics meets time series analysis: unraveling microbial community dynamics \cite{Faust2015}

%-----------------------
% Co-occurrence (geographical, ecological)
%-----------------------

\noindent\rule{\textwidth}{0.4pt}

wiki:
``Ecology is the branch of biology which studies the interactions among organisms and their environment.''

dependencesn not only with meta-data but between species: co-occurrences

\cite{Lima-Mendez2015,Villalobos2017,Zelezniak2015}

might be interesting to visualize co-occurrences with respect to phylogenetic information


%-----------------------
% Medicine and Health Care
%-----------------------

\noindent\rule{\textwidth}{0.4pt}

medicine, human health, health care

was already used in BV

place lineages etc of HIV

in recent years, there have been studies that conduct
phylogenetic analyses of widely researched diseases such as HIV/AIDS \cite{Castro-Nallar2012,Brenner2013}
and different types of cancer \cite{Brown2017a,Abbosh2017}

strain detection: which one is it?!

hmp gut biomonitoring
\acf{HMP} \citep{Huttenhower2012,Methe2012}
that we used in chapters...

in conclusion, there is much to do!
