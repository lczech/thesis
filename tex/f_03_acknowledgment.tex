% \pdfbookmark{Acknowledgments}{Acknowledgments}
% \section*{Acknowledgments}
% \vspace*{1em}

\pdfbookmark{Acknowledgments}{Acknowledgments}
\chapter*{Acknowledgments}
\markboth{Acknowledgments}{Acknowledgments}

\epigraph
{\textit{``Evolution forged the entirety of [...] life on this planet \\ using only one tool -- the mistake.''}}
{--- Dr. Robert Ford (Anthony Hopkins),\\ \textit{Westworld, Season 1: The Original}}

Trying new things, making mistakes, and thus ``to err forward'' are important parts of the scientific method.
% While I tried and learned new things in the course of this work, I hopefully did not make too many mistakes.
In the course of this work, I tried and learned quite a few new things;
% there hence might be mistakes.
I am deeply grateful that I had the opportunity to make all the mistakes that come along with this.
This would not have been possible without the support of a whole lot of marvelous people,
to whom I wish to express my gratitude here.

First and foremost, I want to thank Prof. Alexandros Stamatakis for his excellent scientific supervision,
for sharing his expertise, and for providing me with the freedom to grow and learn.
It is rare to find an advisor who is so dedicated to guiding and supporting his students,
while also being approachable on a human level.

Second, I am grateful to Prof. Emmanuel M\"uller, who agreed to be my second advisor and reviewer of this thesis.
His support and interest in my work, as well as his contributions of ideas, helped to shape this work,
and were the basis for many of the results presented here.

Furthermore, I wish to thank my colleagues at the Exelixis Lab for their support, for many hours of valuable discussions,
be it at the white board or in private, and for making my time in the lab as enjoyable as it was:
Jiajie Zhang, Tom{\'{a}}{\v{s}} Flouri, Paschalia Kapli, Andre Aberer, Kassian Kobert, Diego Darriba, Alexey Kozlov,
Sarah Lutteropp, Benoit Morel, Rudolf Biczok, Dora Serdari, and Ben Bettisworth.
In particular, I want to thank Pierre Barbera, with whom I shared a great many coffee breaks and other occasions
to discuss software design questions, method development strategies, and also personal matters;
without him, this work and my software would be of far inferior quality.

Similarly, I appreciate the scientific and non-scientific exchange I had with the people at the institute, namely,
Kira Feldmann, Benjamin Heinzerling, Johannes Wagner, and Johannes Resin,
as well as Christian Goll, Bernd Doser, Thomas Rasem, and Frauke Bley.
Some of the most fruitful discussions were with Nikos Gianniotis and Kashif Sadiq;
I really hope our collaborative ideas work out one day.
Moreover, I am happy to have shared my workplace and to exchange ideas with the visitors in our lab,
Mark Holder, Emily McTavish, Rebecca Harris, Nikos Psonis, Mourad Elloumi, Khouloud Madhbouh, and Laura Rubinat-Ripoll.

I am also thankful to my collaborators and scientific colleagues, from whom I learned a lot and which helped
conducting and shaping my research, in particular,
Micah Dunthorn, Fr\'{e}d\'{e}ric Mah\'{e}, David Bass, C\'{e}dric Berney, Colomban de Vargas, and Jaime Huerta-Cepas,
but also Javier del Campo, Pelin Yilmaz, Christian Quast, Guillaume Lentendu, Torbj\o{}rn Rognes, Mahwash Jamy,
Antonis Rokas, and Xiaofan Zhou.
There were also several other researchers who kickstarted many ideas of this work and offered their help and advice when needed,
particularly, Alex Washburne, Justin Silverman, Michael Robeson, Sujatha Srinivasan, Frederick Matsen,
Gavin Douglas, and Lionel Guidi; thank you for your inspiration and initiative.
I am also happy that my paths crossed with Nick Goldman, Adam Leach\'{e}, Brian Moore, Ben Redelings, Asif Tamuri,
James Pease, Ziheng Yang, Emmanouela Karameta, David Matten, and Sandra Alvarez Carretero,
and all the other instructors and participants at the Computational Molecular Evolution courses.

On a more personal note, I want to thank my family, my parents Peter and Maria and my sister Judith,
who not only constantly supported me during this thesis,
but through all of my years of study in Karlsruhe, Heidelberg and around the world.
Also, I want to thank Malina Graf and Julia Klawitter, as well as my friends in Heidelberg and all other places,
for their support and understanding.
More thanks go to David Dao, who pointed out the opportunity to work at the institute to me,
as well as Andreas Veit, who is a scientific inspiration to me.
I also want to thank the people of the Heidelberg Unseminars in Bioinformatics for the opportunities they gave me.

Furthermore, I wish to thank Richard Dawkins, whose books inspired me at the exactly right time in my life,
as well as Phil Collins and Peter Gabriel, who provided the background music for this thesis.
A special thanks to \texttt{xkcd} and Randall Munroe for enlightening my journey through academia.
I feel I must also thank the open source software community,
as well as the people and journals supporting open access publication;
you help to put knowledge and science where it belongs: in the public hand.
We can only stand on the shoulders of giants if those are not protected by pay-walls.

Finally, I would like to to express my gratitude towards Klaus Tschira, the Klaus Tschira Stiftung,
and in particular, the Heidelberg Institute for Theoretical Studies,
both for funding my position as well as providing an excellent, rewarding, and fun work environment.
It was a pleasure to work with and alongside so many lovely people.

% thankful that my paths crossed with 

% thanks also to the (mostly) anonymous reviewers of our publications, [cite cite cite]
% who pointed out flaws to us and thus helped to make them better.
% thanks to hub meetings

% not to forget:
% \url{https://xkcd.com/1706/}
% or maybe:
% \url{https://www.xkcd.com/1840/}
% or better:
% \url{https://xkcd.com/1605/}

% thanks open access for making knowledge a public good, for putting it where it belongs: in the public hand.
% we can only stand on the shoulders of giants if they are not protected by pay-walls.
% hence also:
% anti-ack for Elsevier and Springer for being unscientific and opposing the progress of science
% by hindering the access to scientific publications due to their capitalistic/greedy needs and pay-walls.
% making billions from selling articles back to the research community that put money and effort into producing them.
% open science, knowledge and science are for everyone, should be public.
% and the improvement of the state of world as a whole.

% Andre:
%
% First and foremost, I would like to thank Prof. Alexandros Stamatakis for several years
% of excellent scientific supervision, for sharing his expertise and for providing me with
% guidance and freedom in equal parts for accomplishing this thesis.
% I am grateful to Prof. Bernhard Misof for his interest in my research and for agreeing
% to review this thesis. Furthermore, I would like to express my sincere gratitude to
% Prof. Fredrik Ronquist for hosting my research stay at the Naturhistoriska riksmuseet in
% Stockholm, many discussions on Bayesian matters and the intensive research experience
%  ̈
% on Oland.
% On a personal note, I am thankful for the consistent support of my parents Jakob and
% Anita and my brother Dominik. In particular, my deepest gratitude goes to Sabrina
% with whom I could share the ups and downs of my scientific development like with no
% other person.
% As a research venue, the Exelixis lab excels by bringing together a talented group
% of people with diverse backgrounds. Thus, a thanks for many hours of interesting dis-
% cussions and collaboration goes to Fernando Izquierdo-Carrasco, Simon Berger, Nikos
% Alachiotis, Pavlos Pavlidis, Kassian Kobert, Jiajie Zhang, Solon Pissis, Tom ́aˇs Flouri,
% Diego Darriba, Paschalia Kapli, Alexey Kozlov and Lucas Czech. Moreover, I am happy
% that my paths crossed with Mark Holder, Emily McTavish, Will Pearse and Christian
% Goll.
% Finally, I would like to express my gratitude to Klaus Tschira, founder of the Heidelberg
% Institute for Theoretical Studies which provided the funding for my position.

% Alexey:
%
% Above all, I want to thank my supervisor, Prof. Dr. Alexandros Stamatakis for his
% invaluable assistance. He was always extremely supportive in both scientific and
% non-scientific matters, and I felt lucky to work under his guidance.
% I am furthermore grateful to my co-advisor Prof. Dr. David Posada, who kindly
% agreed to review this thesis. I also appreciate our ongoing collaboration on cancer
% cell phylogeny inference, and I would like to thank Prof. Posada for hosting me
% during my research visit to Vigo.
% My former and current colleagues Fernando Izquierdo-Carrasco, Jiajie Zhang,
% Tomas Flouri, Paschalia Kapli, Andre Aberer, Kassian Kobert, Diego Darriba, Lu-
% cas Czech, Pierre Barbera, Sarah Lutteropp, Benoit Morel, Rudolf Biczok and Dora
% Serdari were always friendly, collaborative and willing to share their knowledge. I
% always enjoyed spending my time with them, be it in the lab, on the Neckarwiese, or
% at one of the inventive cocktail parties organized by Lucas. I was also happy to share
% my workplace and to exchange ideas with our short- and long-term visitors Mark
% Holder, Emily Jane McTavish, Rebecca Harris, Nikos Psonis, Khouloud Madhbouh,
% and Laura Rubinat. I am grateful to my collaborators Pelin Yilmaz, Karen Meuse-
% mann, Ralph Peters, Oliver Niehuis, Manuela Sann, Sara Bank, Oliver Ratmann,
% and Micah Dunthorn for productive teamwork and interesting discussions. Further-
% more, I would like to express my gratitude to Nick Goldman, Adam Leache and
% Deren Eaton for their helpful hints with respect to sequence uncertainty modeling.
% Beyond work, I am sincerely thankful to all my friends from Karlsruhe, Hei-
% delberg and Ladenburg, who accompanied me in countless leisure activities during
% these four years. In particular, I would like to acknowledge Felix and Dasha for be-
% ing my faithful team fellows in the ’What?Where?When?’ quiz game, and also for
% giving me (maybe a bit too) ample diversion opportunities during the final writing
% phase. Speaking of which, nothing helped me to concentrate on the writing like the
% music of God is an Astronaut. And Scar Symmetry (among many other great metal
% bands) yielded a speedup of at least 2× for coding and cluster job submission tasks.
% Finally and importantly, I am grateful to the Klaus Tschira Foundation for
% funding my position, and to the Heidelberg Institute for Theoretical Studies for
% providing an excellent work environment.

% Thanks to all of you!

% Widmungen
% • Danksagungen
% • Persönlicher Bezug zum Thema
% • Nennung und Würdigung der externen Institutionen
% • und den entsprechenden Ansprechpartnern vor Ort

% \begin{center}
% 	\includegraphics[width=1.0\textwidth,keepaspectratio=true]{./logos/bild_logo_bwsbwstiftung.jpg}
% \end{center}
% \vspace{-6cm}

% \vspace*{3em}
% \vfil
% \hspace{2.2cm}\includegraphics[width=0.4\textwidth,keepaspectratio=true]{./logos/interACT_pix.jpg}
% \hspace{2.3cm}\includegraphics[width=0.4\textwidth,keepaspectratio=true]{./logos/logo_interact_4c2.pdf}
% \vspace{0cm}
% \begin{center}
% 	\includegraphics[width=1.0\textwidth,keepaspectratio=true]{./logos/B_ed1cad28a0.png}
% \end{center}

\nocite{BW2019}
