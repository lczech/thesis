% ######################################################################################################################
%         Pipeline and Implementation
% ######################################################################################################################

\cleardoublepage

\chapter{Pipeline and Implementation}
\label{ch:PipelineImplementation}

\paperbox{
    This chapter is based on the peer-reviewed publications:
}{\paperart \paperpppp}{
    \textbf{Contributions:} Lucas Czech... Alexandros Stamatakis...
}

\todo{rename chapter to ``Genesis and Gappa''?}

\todo{add genesis and gappa logos. because we can.}

\todo{maybe post the whole draft of the genesis/gappa paper here?}

\todo{maybe add the pipleline and other supporting texts from the papers here}

The methods described here are implemented in our tool \toolname{gappa},
which is freely available under GPLv3 at \url{http://github.com/lczech/gappa}.
\toolname{gappa} internally uses our C++ library \toolname{genesis},
which offers functionality for working with phylogenies and phylogenetic placement data,
and also contains methods to work with taxonomies, sequences and many other data types.
\toolname{genesis} is also freely available under GPLv3 at \url{http://github.com/lczech/genesis}.

\toolname{gappa} offers a command line interface for conducting typical tasks when working with phylogenetic placements.
The methods that we described here are implemented via the following sub-commands:

\begin{itemize}
    \item \texttt{dispersion}: The command takes a set of jplace files (called samples), and calculates and visualizes
        the Edge Dispersion per edge of the reference tree.
    \item \texttt{correlation}: The command takes a set of jplace samples, as well as a table containing metadata
        features for each sample. It then calculates and visualizes the Edge Correlation with the metadata features per
        edge of the reference tree.
    \item \texttt{phylogenetic-kmeans} and \texttt{imbalance-kmeans}: Performs $k$-means clustering of a set of jplace
        files according to our methods.
    \item \texttt{squash} and \texttt{edgepca}: Reimplementations of the two existing methods \cite{Matsen2011a,Evans2012}.
\end{itemize}

These are the \toolname{gappa} commands that are relevant for this paper.
The tool also offers additional commands that are useful for phylogenetic placement data, such as visualization or filtering.
At the time of writing this manuscript, \toolname{gappa} is under active development,
with more functions planned in the near future.
Lastly, we provide prototype implementations, scripts, data, and other tools
used for the tests and figures in this paper at \url{http://github.com/lczech/placement-methods-paper}.


\todo{ART:}

An implementation of the methods described here is freely available in our tool \toolname{gappa},
which is published under GPLv3 at \url{http://github.com/lczech/gappa}.
\toolname{gappa} is based on our C++ library \toolname{genesis},
which offers functionality concerning phylogenies and phylogenetic placement data,
but also has functions to work with sequences, taxonomies and many other data types.
\toolname{genesis} is also published under GPLv3 and is available at \url{http://github.com/lczech/genesis}.

\toolname{gappa} offers a command line interface for typical tasks of working with phylogenetic placements.
The methods described in this paper are implemented via four sub-commands:

\begin{itemize}
    \item \texttt{phat}: Phylogenetic Automatic (Reference) Tree method.
          The command expects a taxonomy file and a sequence file of a sequence database,
          e.g., \toolname{Silva} \citep{Quast2013,Yilmaz2014},
          as well as the target number of consensus sequences to be generated for the intended phylogeny.
          The result is a \texttt{fasta} file with consensus sequences representing taxonomic clades.
          The command can be further customized, e.g., by changing the consensus sequence method,
          using only a specified subclade of the taxonomy for running the algorithm,
          as well as several detail settings for the method.
          It can also output additional info files that allow to inspect details of the calculations,
          like the number of sequences and their entropy per clade.
    \item \texttt{extract}: Extract/collect placements in specific sub-clades of the tree.
          The command performs the main step of the multilevel placement approach.
          Its input is a set of \texttt{jplace} files containing placements on the backbone tree,
          as well as a file listing the clade name that each taxon of the backbone tree belongs to.
          For each clade, it then writes a new \texttt{jplace} file,
          containing all queries that were placed in that clade with more than a customizable threshold
          of their placement mass.
          \\
          Furthermore, if provided with the sequence files that were used to make the input \texttt{jplace} files,
          the corresponding sequence of each query are also written to \texttt{fasta} files per clade.
          Thus, a per-clade collection of sequences is created, where each result file contains the sequences
          that were placed in this clade of the backbone tree.
          These can then be used for the second level placement on separate clade-specific trees.
    \item \texttt{chunkify}: Split a set of \texttt{fasta} files into chunks of equal size,
          and write abundance maps.
          The command re-names the sequences using a configurable hash function (MD5, SHA1 or SHA256),
          and de-duplicates across all input sequences.
          Its output are chunk files of sequences, as well as an abundance map file for each input sequences file.
          The sequence chunk files can then be used to perform phylogenetic placement
          to obtain per-chunk \texttt{jplace} files.
    \item \texttt{unchunkify}: Take the per-chunk \texttt{jplace} files as well as the abundance map files,
          and generate a \texttt{jplace} for each original sequence file, including the correct abundances.
          This command is the second step of the \texttt{chunkify} command, and reverts its effect,
          so that the resulting \texttt{jplace} files are as if they were created using the original sequence files.
    \item \texttt{assign}: Perform taxonomic assignment using phylogenetic placements.
          While this is not the main focus of this work, we briefly introduce this method here.
          The command uses a taxonomic labeling of the tips of the reference tree
          to annotate all inner branches of the tree with the longest common taxonomic label
          for the induced subtree of the inner branch, in analogy to \toolname{Sativa} \citep{Kozlov2016}.
          Then, each query sequence in the provided \texttt{jplace} files
          is taxonomically assigned according to the labels of the branches where it does have placement mass.
          This can subsequently either be used for taxonomic assignment of the query sequences themselves,
          or to obtain a taxonomic profile of one or more samples.
\end{itemize}

These are the commands of \toolname{gappa} relevant for this paper,
but it also offers more commands that are useful when working with phylogenetic placements.
For details on the commands, and additional potentially useful commands,
see the \toolname{gappa} documentation at \url{https://github.com/lczech/gappa/wiki}.
At the time of writing, it is under active development, and more functions are planned for the near future.
Furthermore, we provide prototype implementations, scripts, data and other tools
used for the tests and figures in this paper at \url{http://github.com/lczech/placement-methods-paper}.
