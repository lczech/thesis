% ------------------------------------------
%  Text editing
% ------------------------------------------

\newcommand\todo[1]{{\color{purple}{TODO: #1}}}

% ------------------------------------------
%  Custom styles
% ------------------------------------------

\newcommand\toolname{\textsc}
\newcommand\taxonname{\textit}
\newcommand\fileformat{\texttt}
\newcommand\nucleobase[1]{\texttt{#1}}

% Epigraph
% https://tex.stackexchange.com/questions/193178/specific-epigraph-style
\usepackage{epigraph}
\setlength\epigraphwidth{.8\textwidth}
\setlength\epigraphrule{0pt}

% ------------------------------------------
%  References
% ------------------------------------------

\newcommand\secref[1]{Section~\ref{#1}}
\newcommand\chpref[1]{Chapter~\ref{#1}}
\newcommand\appref[1]{Appendix~\ref{#1}}

\newcommand\figref[1]{Figure~\ref{#1}}
\newcommand\tabref[1]{Table~\ref{#1}}
\newcommand\eqnref[1]{Equation~\ref{#1}}
\renewcommand\algref[1]{Algorithm~\ref{#1}}

% \newcommand\suppfigref[1]{S\ref{#1}~Fig}

% ------------------------------------------
%  abbreviations
% ------------------------------------------

\def\eg{e.\,g.\ }
\def\ie{i.\,e.\ }
\def\cf{c.\,f.\ }

\DeclareSIUnit\basepair{bp}

% Remove hyperlink color boxes from acronyms.
% https://tex.stackexchange.com/questions/73063/acronym-hyperlink-without-special-color/73068
\makeatletter
\AtBeginDocument{%
  \renewcommand*{\AC@hyperlink}[2]{%
    \begingroup
      \hypersetup{hidelinks}%
      \hyperlink{#1}{#2}%
    \endgroup
  }%
}
\makeatother

% ------------------------------------------
%  Math operators
% ------------------------------------------

\def\avg{\mathop{\mathgroup\symoperators avg}}

% https://tex.stackexchange.com/a/5255
\DeclareMathOperator*{\argmax}{arg\,max}
\DeclareMathOperator*{\argmin}{arg\,min}

% ------------------------------------------
%  Page Handling
% ------------------------------------------

% Leerseite, nächste Seite rechts
\newcommand{\blankpage}{
 \clearpage{\pagestyle{fancy}\cleardoublepage}
}

% marker for lowercase references
% (see http://tex.stackexchange.com/questions/1655/correct-case-in-namerefs)
\newcommand{\mlr}{}

% use marker to define lowercase reference command...
\newcommand{\lnameref}[1]{%
\bgroup
\let\mlr\MakeLowercase
\nameref{#1}\egroup}

% ... and first upper, rest lowercase command
\newcommand{\fnameref}[1]{%
\bgroup
\def\mlr{\let\mlr\MakeLowercase}%
\nameref{#1}\egroup}

% ------------------------------------------
%  Papers References and Boxes
% ------------------------------------------

% ART Paper
\newcommand\paperart{
    \item \textbf{Lucas Czech}, Pierre Barbera, and Alexandros Stamatakis.
        "Methods for Automatic Reference Trees and Multilevel Phylogenetic Placement."
        \textit{Bioinformatics}, 2018.
}

% Phylogenetic Postprocessing Paper
\newcommand\paperpppp{
    \item \textbf{Lucas Czech} and Alexandros Stamatakis.
        "Scalable Methods for Post-Processing, Visualizing, and Analyzing Phylogenetic Placements."
        \textit{PLOS One}, 2018.
}

% Command for framed boxes.
\newcommand\paperbox[3]{
    \begin{framed}
    % \noindent
    #1
    \begin{itemize}
        #2
    \end{itemize}
    % \noindent
    #3
    \end{framed}
}
